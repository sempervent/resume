\documentclass[10pt]{article}%
\author{Joshua N Grant}
\title{CV for Joshua N. Grant}
%\usepackage[backend=biber,style=authoryear,sorting=nyt,natbib=true,maxbibnames=99,firstinits=true,uniquename=false]{biblatex}
\usepackage{etoolbox}
\patchcmd{\thebibliography}{\section*{\refname}}{}{}{}
\usepackage{tikz}
\usepackage[hidelinks]{hyperref}
\usepackage{xcolor}
\usepackage[left=1cm,top=1cm,right=1cm,bottom=1cm,nohead,nofoot]{geometry}
\usepackage{tabularx}
\usepackage{makecell}
\usepackage{palatino}
\usepackage{pifont}
\usepackage{enumitem}
\usepackage{sectsty}
\usepackage{fontawesome}
\hypersetup{colorlinks=false}
%\addbibresource{bibliography.bib}
\bibliographystyle{plain}
\definecolor{fillheader}{HTML}{FFFFFF}
\definecolor{header}{HTML}{000000}
\definecolor{headercolor}{HTML}{e3e3e3}
\definecolor{linkcolor}{HTML}{3e3e3e}
\usetikzlibrary{positioning,shapes,shadows,arrows,backgrounds,calc,shadows.blur}
\newcolumntype{M}{>{\raggedright\arraybackslash}X}
\setlength{\parindent}{0pt}
\sectionfont{\fontsize{10}{11}\selectfont}
\begin{document}
\pagenumbering{gobble}
\begin{minipage}[ht]{.33\linewidth}%
  \setlength{\headsep}{-10pt}%
  \setlength{\voffset}{-0.75in}%
  {\Large Joshua N.\huge\textbf{Grant}} \\
   \large \textit{Data Engineer}
\end{minipage}
\begin{minipage}[ht]{.25\linewidth}%
  \section*{\faUser{} Contact}%
  12918 Sanderling Lane \\
  Knoxville, TN 37922 \\
  United States of America \\
  \textbf{+1 (865) 803 3495} \faMobile{} \\
\end{minipage}
\begin{minipage}[ht]{.5\linewidth}%
  \href{mailto:jngrant@live.com}{\color{linkcolor}{jngrant@live.com \faEnvelope}} \\
  \href{https://notjustadatum.blogspot.com}{\color{linkcolor}{notjustadatum.blogspot.com \faBold}} \\
  \href{https://github.com/sempervent}{\color{linkcolor}{github.com/sempervent \faGithubAlt}} \\
  \href{https://www.linkedin.com/in/joshuanagrant}{\color{linkcolor} {\small linkedin.com/in/joshuanagrant} \faLinkedin} \\
  \href{https://scholar.google.com/citations?user=vs-HJQcAAAAJ&hl=en}{\color{linkcolor} {scholar.google.com \faPencil}}
\end{minipage}
\par\noindent\rule{\textwidth}{0.4pt}
\begin{center}
\section*{\faBullseye{} Objective}%
To take theories, ideas, research, and experiments and turn them into production-scalable solutions and to develop and support Python-based applications and data workflows on cloud infrastructure
\end{center}
\par\noindent\rule{\textwidth}{0.4pt}
\begin{minipage}[t]{.25\linewidth}%
  \flushleft{}
   \section*{\faCode{} Coding}%
     {\small
        \begin{tabularx}{\linewidth}{p{.44\linewidth}X}
           \textbf{Bash}:       & \ding{110} \ding{110} \ding{110} \ding{110} \color{headercolor}{\ding{110}} \\
           \textbf{Docker}:     & \ding{110} \ding{110} \ding{110} \ding{110} \color{headercolor}{\ding{110}} \\
           \textbf{Kubernetes}: & \ding{110} \ding{110} \color{headercolor}{\ding{110} \ding{110} \ding{110}} \\
           \textbf{HTML/CSS}:   & \ding{110} \ding{110} \ding{110} \color{headercolor}{\ding{110} \ding{110}} \\
           \textbf{JavaScript}: & \ding{110} \ding{110} \color{headercolor}{\ding{110} \ding{110} \ding{110}} \\
           \textbf{\LaTeX}:     & \ding{110} \ding{110} \ding{110} \color{headercolor}{\ding{110} \ding{110}} \\
           \textbf{Perl}:       & \ding{110} \color{headercolor}{\ding{110} \ding{110} \ding{110} \ding{110}} \\
           \textbf{Go}:         & \ding{110} \color{headercolor}{\ding{110} \ding{110} \ding{110} \ding{110}} \\
           \textbf{PHP}:        & \ding{110} \ding{110} \color{headercolor}{\ding{110} \ding{110} \ding{110}} \\
           \textbf{Python}:     & \ding{110} \ding{110} \ding{110} \ding{110} \color{headercolor}{\ding{110}} \\
           \textbf{R}:          & \ding{110} \ding{110} \ding{110} \ding{110} \color{headercolor}{\ding{110}} \\
           \textbf{SQL}:        & \ding{110} \ding{110} \ding{110} \ding{110} \color{headercolor}{\ding{110}} \\
           \textbf{git}:        & \ding{110} \ding{110} \ding{110} \ding{110} \color{headercolor}{\ding{110}} \\
           \textbf{CI/CD}:      & \ding{110} \ding{110} \color{headercolor}{\ding{110} \ding{110} \ding{110}} \\
           \textbf{AWS}:        & \ding{110} \ding{110} \color{headercolor}{\ding{110} \ding{110} \ding{110}}
        \end{tabularx}
     }
\end{minipage}
\begin{minipage}[t]{.35\linewidth}%
\flushleft
\section*{\faCubes{} Flavors}%
   {\small
      \begin{tabularx}{\linewidth}{p{.2\linewidth}XX}
         \textbf{R}: & data.table & shiny \\
                     & tidyverse  & vegan \\
                     & testthat   & devtools \\
                     & DBI        & leaflet \\
                     & httr       & ggplot2 \\
         \hline
         \textbf{Python}: & pandas & bs4 \\
                          & PIL    & xdgboost \\
                          & osgeo  & airflow \\
                          & selenium & geopandas \\
         \hline
         \textbf{SQL/DB}: & MySQL & PostgreSQL \\
                       & MongoDB & Hadoop \\
                       & SQLite  & Snowflake \\
                       & ElasticSearch & \\
         \hline
         \textbf{\LaTeX }:  & tikz   & fancyhdr \\
                           & tables & booktabs \\
         \hline
         \textbf{JavaScript}: & node.js & angular.js \\
                              & vue.js & jQuery
     \end{tabularx}}
     \flushleft%
\end{minipage}
\begin{minipage}[t]{.4\linewidth}%
   \section*{\faBarChart{} Statistics}%
   \flushright
   {\small
      \begin{tabularx}{\linewidth}{X}
        \textbf{ANOVA} \\
        \textbf{Linear Regression} \\
        \textbf{PCA/PCoA} \\
        \textbf{Markov Chain} \\
        \textbf{Clustering} \\
        \textbf{Exploratory Data Analysis} \\
        \textbf{Dynamic Time Warping} \\
        \textbf{Self-Organizing Maps} \\
        \textbf{Taxonomic Classifications} \\
      \end{tabularx}}
      \flushleft%
\end{minipage}
\par\noindent\rule{\textwidth}{0.4pt}
\section*{\faCalendar{} Experience}
   \begin{tabularx}{\linewidth}{M}%
      \makecell[l]{\textbf{Data Engineer}} \\
      Oak Ridge National Laboratory, Oak Ridge, TN \textit{2019\textemdash Present} \\
      \begin{itemize}[topsep=-12pt,parsep=0em]
          \setlength\itemsep{0em}
          \item Serves as support staff for the Python project EAGLE-I, where I implemented an ETL workflow from a brittle-based cron system to a more robust Airflow system deployed in a Kubernetes cluster and in the cloud %
          \item Developed full stack docker images using Python, R, Airflow,  and Shiny to automate the acquisition of data for the WSTAMP project for storage in an Hadoop Data Warehouse %
          \item Led the development team for DOE's COVID-19 tracking project using both R and Python and downstream model creation and analysis, for which I received distinguished recognition for the project %
          \item Developed containerized ETL workflows, Python microservices, and REST APIs using Swagger for data exchange in the NAERM project %
          \item Automated web-scraping and social media ingestion via containerized model creation for misinformation classification for implementation in advanced statistical models and network analysis and ingestion into ElasticSearch %
          \item Serves as Software Quality Assurance board chair for the Geospatial Science and Human Security Division %
      \end{itemize} \\
           \makecell[l]{\textbf{Data Scientist II Subcontractor}} \\
      Oak Ridge National Laboratory, Oak Ridge, TN \textit{2018 \textemdash 2019} \\
      \begin{itemize}[topsep=-12pt,parsep=0em]
          \setlength\itemsep{0em}
          \item Automated data ingestion of the WSTAMP Web Application via the use of R, Python, Airflow, and Docker %
          \item Authored an R package to unify geography between online data APIs %
          \item Developed a SpatiaLite SQLite database to ease lookup of geographic data %
          \item Developed a zonal statistics application in Python that takes a raster image and shapefile and computes the statistics over the shapes stored in the shapefile %
        \end{itemize} \\
   \end{tabularx}
   \newpage
   \begin{tabularx}{\linewidth}{M}%
         \makecell[l]{\textbf{Owner \& Contractor-For-Hire}} \\
         Specrabella, Knoxville, TN \textit{2018 \textemdash 2018 } \\
      \begin{itemize}[topsep=-12pt,parsep=0em]
         \setlength\itemsep{0em}
         \item Designed, developed, and deployed an interactive energy informatics web application that provides real-time calculation from machine learning derived models %
         \item Consulted with clients on ground-up web application architecture, database and visualization needs%
          \item Developed data ingestion methods using cURL, PHP, and R to scrape energy data from the web % 
          \item Streamlined training and development procedures using ansible, vagrant, PHP, and BASH % 
      \end{itemize} \\
      \makecell[l]{\textbf{Bioinformatician, Project Manager, \& Technical Sales Representative}} \\
      Microbial Insights, Knoxville, TN \textit{2016 \textemdash 2018} \\
      \begin{itemize}[topsep=-12pt,parsep=0em]
         \setlength\itemsep{0em}
         \item Established an automated pipeline using R, MySQL, Python, and Bash for NGS analysis, along with intranet website control written in JavaScript and Shiny  %
         \item Automated client-bound statistical calculations such as linear models, ANOVA, SOMs, clustering, PCA, and PCoA %
         \item Developed an ETL customer data visualization tool using R, PHP, JQuery, and MySQL to view qPCR results in the context of other samples' and selected parameters %
         \item Optimized data delivery to clients via a custom R package and local shiny applications to quickly generate \LaTeX{} PDF reports  %
         \item Constructed, populated, and maintained an intranet wiki based using PostgreSQL and PHP to aid in project management and customer service
        \end{itemize} \\
       \makecell[l]{\textbf{Graduate Research Assistant, Dr. Neal Stewart's Plant Biotechnology Lab}} \\
       University of Tennessee, Knoxville, TN \textit{2014 \textemdash{} 2016} \\
       \begin{itemize}[topsep=-12pt,parsep=0em]
            \setlength\itemsep{0em}
            \item Summarized statistical findings of cell suspensions using linear models, ANOVA, and PCA in Python and R %
            \item Evaluated suspension cultures via chemical and spectral processes for lignin formation and statistically analyzed and summarized my findings for inclusion in DOE reports %
            \item Collaborated on a novel single cell suspension and cryopreservation robotic system %
            \item Advanced and executed monocot genetic modifications for \textit{Panicum} spp., \textit{Oryza} spp., \textit{Sorghum} spp., \textit{Saccharum} spp., \textit{Zea mays}
         \end{itemize} \\
       \makecell[l]{\textbf{Laboratory Assistant, Dr. Neal Stewart's Plant Biotechnology Lab}} \\
       University of Tennessee, Knoxville, TN \textit{2012 \textemdash{} 2014} \\
       \begin{itemize}[topsep=-12pt,parsep=0em]
            \setlength\itemsep{0em}
            \item Developed automated statistical methodology for screening of lignin content and imaging of cell characteristics both \textit{in vivo} and \textit{in vitro} using Python%
            \item Extracted genomes from NCBI, cleaned and normalized the data, and performed exploratory data analysis using Python
            \item Complied with USDA-APHIS regulations regarding transgenic plants
            \item Implemented an \textit{E. coli} bioreactor for production of proteins
         \end{itemize} \\
        \makecell[l]{\textbf{Laboratory Assistant, Dr. Paris Lambdin's Biosystematics and Biological Control Lab}} \\
        University of Tennesse, Knoxville, TN \textit{2002} \textemdash{} 2012 \\
        \begin{itemize}[topsep=-12pt,parsep=0em]
        	    \setlength\itemsep{0em}
        	    \item Designed instructional modules for undergraduate and graduate level courses using HTML, CSS, JavaScript, and Flash
        	    \item Collected samples and data from field locations
        	    \item Assisted in community outreach programs such as Bloomsdays, Buggy Buffet, and 4-H Camps
        	    \item Performed forest coverage analysis using ArcGIS
        \end{itemize}
    \end{tabularx}
\par\noindent\rule{\textwidth}{0.4pt}
\section*{\faUniversity{} Education}
     \begin{tabularx}{\linewidth}{M}% education
      \makecell[l]{\textbf{Master of Science in Plant Sciences \textemdash Plant Molecular Genetics}} \\
      University of Tennessee \textemdash ~Knoxville, TN \textemdash ~\textit{Spring 2017} \textemdash ~\textbf{GPA: 3.72/4.0} \\
      \\
      \makecell[l]{\textbf{Bachelor of Science in Plant Sciences \textemdash Biotechnology}} \\
      University of Tennessee \textemdash ~Knoxville, TN \textemdash ~\textit{Spring 2014} \textemdash ~\textbf{GPA: 3.74/4.0} \\
      \textbf{Magna Cum Laude} \\
    \end{tabularx}%
\newpage
\par\noindent\rule{\textwidth}{0.4pt}
\section*{\faBook{} Publications}
\nocite{*}
\bibliography{bibliography}
\par\noindent\rule{\textwidth}{0.4pt}
\section*{\faUserPlus{} References}
References available upon request.
\end{document}
